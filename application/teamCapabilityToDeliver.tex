% include Joaquin, Nick, Maneesh, Tiago and Goncalo

\subsection{Contributions to the generation of new ideas, tools, methodologies, or knowledge}

% Joaquin (funding): UCSD
% Joaquin, Goncalo (funding): BBSRC23 grant Joaquin,
% Goncalo, Nick (tools): BonsaiML
In 2021 Dr.~Rapela and Dr.~Lopes generated the main ideas of the BBSRC funded
award
\href{https://gow.bbsrc.ukri.org/grants/AwardDetails.aspx?FundingReference=BB\%2FW019132\%2F1}{Machine
learning for neuroscience experimental control} and in collaboration with
Dr.~Guilbeault they are now building the
\href{https://github.com/bonsai-rx/machinelearning}{Bonsai.ML} package, which
adds machine learning functionality to the
\href{https://bonsai-rx.org/}{Bonsai} ecosystem.

In 2013 Dr.~Rapela was awarded a seed grant from the Center for Brain Activity
Mapping of the Kavli Institute for Brain and Mind at UC San Diego titled
\emph{Taking the next step toward understanding computations by neural
ensembles with high resolution neural recordings, generic data assimilation
methods, and increased computational power}.

% Joaquin (knowledge): EEG and autism
% Joaquin (knowledge): brain oscillations and EEG
% Joaquin (knowledge): travelling waves in speech production
% Joaquin (knowledge): intracortical recordings and epilepsy
Dr.~Rapela uses advanced statistical methods to understand brain function. He has
generated knowledge on the estimation of receptive fields of visual neurons
stimulated with natural images~\citep{rapelaEtAl06,rapelaEtAl10}, on the use of
the electroencephalogram to understand cognition and mental disorders in
humans~\citep{rapelaEtAl12-attentionSwitch,rapelaEtAl12-eyeTracking,rapelaEtAl18-avshift},
on the use of nonlinear dynamical systems to model neural population
activity~\citep{rapelaEtAlInPrepEDMs}, on the use of surface cortical
recordings in humans to understand the relation between brain oscillations and
speech~\citep{rapelaInPrepTWsInSpeech,rapelaInPrepSyncTWs,rapelaInPrepSyncTWsII}
and on the use of intracortical recordings in humans to understand the neural
basis of
epilepsy~\citep{rapelaEtAl19-epilepsy-tsne,rapelaAndTodorov19-epilepsy-hmm}.

% Joaquin statistical neuroscience methods developer
Dr.~Rapela is the lead developer of several open-source statistical algorithms
with applications in neuroscience like
\href{https://github.com/joacorapela/svGPFA}{Sparse Variational Gaussian
Process Factor Analysis},
\href{https://github.com/joacorapela/lds\_python}{Linear Dynamical Systems} and
\href{https://github.com/joacorapela/hiddenMarkovModels}{Hidden Markov Models}.

\subsection{The development of others and maintenance of effective working relationships}

% Joaquin mentoring: Tsong-Yan, Aisha, Zimo
In 2011, Dr.~Rapela has co-mentored Mr.~Tsong-Yan Lin during his MSc studies at
UC San Diego~\citep{rapelaEtAl12-eyeTracking}. In 2021 and 2023 he has mentored
Ms.~Aisha Qureshi and Ms.~Zimo Li, respectively, in seven-month-long
\href{https://www.simonsfoundation.org/grant/shenoy-undergraduate-research-fellowship-in-neuroscience-surfin/}{Shenoy
Undergraduate Research Fellowship in Neuroscience (SURFiN)} fellowships,
sponsored by the
\href{https://www.simonsfoundation.org/collaborations/global-brain/}{Simons
Collaboration on the Global Brain}.

% Joaquin collaborations:
As a computational neuroscientists Dr.~Rapela has build multiple collaborations
with experimental neuroscientists. For example,
%
with Dr.~Jon Touryan and Dr.~Guidon
Felsen~\citep{rapelaEtAl06,rapelaEtAl10}.
%
with Prof.~Jeanne Townsend~\citep{rapelaEtAl12-attentionSwitch,rapelaEtAl12-eyeTracking,rapelaEtAl18-avshift}
and with Prof.~Edward Chang~\citep{rapelaEtAlInPrepEDMs,rapelaInPrepTWsInSpeech,rapelaInPrepSyncTWs,rapelaInPrepSyncTWsII}.


% Joaquin teaching

% Joaquin lead research team CBAM
% Joaquin lead research team BBSRC

\subsection{Ccontributions to the wider research and innovation community}

% Joaquin reviewer
% Joaquin RCWG

\subsection{Contributions to broader research or innovation users and audiences and towards wider societal benefit}

