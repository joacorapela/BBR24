\subsection{Context}

% add references
The use of large amounts of data in image and speech recognition and more
recently in large language models has generated breakthroughs in the
capabilities of machine learning models. Yet, most animal experiments in
Neuroscience still generate limited amounts of data, as animal behaviour is
heavily constrained and the duration of experiments is short.
%
Long-duration experiments, where animals can move freely in naturalistic
environments, combined with advanced machine learning methods, could reveal new
aspects of behaviour and brain function not evident in data generated in simpler
experiments.
%
At the Sainsbury Wellcome Centre (SWC) for Neural Circuits and Behaviour and the
Gatsby Computational Neuroscience Unit (GCNU) we are developing hardware and
software infrastructure to perform long-duration and naturalistic foraging
experiments, as well as creating advanced machine learning methods to control
these experiments and extract insights from the generated data.
%
Here we propose to create a resource to share openly (1) behavioural and
electrophysiological recordings generated by our foraging experiments, (2)
hardware and software infrastructure to build long-duration and naturalistic
experiments, (3) machine learning methods to process the data generated by
these experiments, and (4) provide online support to people using the
distributed data, hardware, software and methods.

% hardware advances
%
% Goncalo please improve this paragraph
% Recent hardware and software developments now enable advanced animal
% experimentation.
% 
% It is now possible to precisely monitor animal behavior with 2D and 3D video
% cameras.
%
% It is also now possible to record spikes of hundreads and thousands of neurons
% simultaneously~\citep{neuropixels}.

% software advances Niko please help improve the following
% Machine learning software now allows sophisticating processing of the output of
% these cameras to, for example, track body parts of single~\citep{deeplabCut}
% and multiple~\citep{sleap} animals, or infer animal behavioral
% syllables~\citep{moseq}.
%
% Real time and close loop complex animal experiments can now be easily developed
% using open source software for visual reactive programming~\citep{bonsai}.
% foraging working group

\subsection{The research the infrastructure, facility or resource will enable}

The hardware, software and user support we propose to distribute will allow
research groups around the world to develop their own long-duration and
naturalistic experiments, enabling a new type of animal experimentation.

These experiments are costly, since they require specialised arenas to house
freely behaving animals for extended periods of time, as well as hardware and
software to monitor their behaviour, to perform neural recording and to store
large amounts of data. For researchers not able to develop their own
experiments, we will share openly behavioural and physiological recordings
from our mouse foraging experiments, enabling them to study long-duration mouse
foraging at the behavioural and physiological levels.

The foraging experiments require behavioural and neural manipulations in real
time, for which online machine learning algorithms are needed. In addition, the
very large size of the datasets generated by these experiments demand novel
machine learning methods. Thus, these foraging experiments generate new demands
of machine learning algorithms that will be addressed with novel machine
learning methods developed by us and/or by machine learning scientists using
the openly shared foraging data.

\subsection{Aims and objectives}

The aims of the proposed resource are:

\begin{enumerate}

    \item share behavioural and electrophysiological recordings from long-duration
        and naturalistic foraging experiments,

    \item share hardware and software specifications to build long-duration
        naturalistic experiments,

    \item share machine learning methods for online and offline processing the
        data generated by these experiments,

    \item provide online advise to scientists interested in use the distributed
        data, hardware, software and methods.

\end{enumerate}

\subsection{Potential user communities, applications and benefits}

% # potential user communities and applications
% . research groups interested in building long-duration and naturalistic experiments
% . ML data scientists
% . business sectors (e.g.):
%   . pharmaceutical companies testing drug side effects on animal models
%   . personalized medicine
%
% # potential benefits
% . novel neuroscience discoveries in the big data regime (e.g., decision making in foraging behavior)
% . novel machine learning algorithms to process online data and very large amounts of batch offline data.
% . business breakthroughs (e.g.):
%   . high through output drug side effect screening
%   . novel patient monitoring methods for personalized medicine

The following user communities could benefit from the proposed resource:

\begin{description}

    \item[research groups able to build long-duration naturalistic experiments]
        could use our distributed resources to build their own experiment and
        generate new scientific discoveries.

    \item[research groups interested in foraging] could use our shared datasets
        to address their questions and generate new findings.

    \item[machine learning scientists] could use our shared datasets to test
        their own algorithm and contribute new methods to model long-duration
        and naturalistic datasets.

    \item[business entities] using long-duration and/or naturalistic
        experiments could benefit from our distributed resources and improve
        their processes. For example, pharmaceutical businesses are starting to
        use whole animal screening to test for side effects on drugs. They
        could use our distributed resources to improve animal behavioral and
        neural monitoring.

\end{description}

