\subsection{Context}

Current animal experimentation in Neuroscience is mostly contrained to simple,
well controlled and short duration experiments. 
%
More sophiticated and long-duration experiments, where animals can move freely
in naturalistic environments, could reveal new aspects of behavior and brain
function not evident in simpler experiments.
%
At the Sainsbury Wellcome Centre (SWC) for Neural Circuits and Behavior and the
Gatsby Computational Neuroscience Unit (GCNU) we are developing hardware and
software infrastucture to perform long-duration and naturalistic foraging
experiments, as well as creating advanced machine learning methods to control
and extract insights from the data generated by these experiments.
%
Here we propose to create a resource to share openly behavioral and
electrophysiolocial recordings generated by our foraging experiments, hardware
and software infrastructure to build long-duration and naturalistic
experiments, and machine learning methods to process the data generated by
these experiments. This resource will also provide online support to people
using the distributed data, hardware, software and methods.

% hardware advances
%
% Goncalo please improve this paragraph
Recent hardware and software developments now enable advanced animal
experimentation.
% 
It is now possible to precisely monitor animal behavior with 2D and 3D video
cameras.
%
It is also now possible to record spikes of hundreads and thousands of neurons
simultaneously~\citep{neuropixels}.

% software advances Niko please help improve the following
Machine learning software now allows sophisticating processing of the output of
these cameras to, for example, track body parts of single~\citep{deeplabCut}
and multiple~\citep{sleap} animals, or infer animal behavioral
syllables~\citep{moseq}.
%
Real time and close loop complex animal experiments can now be easily developed
using open source software for visual reactive programming~\citep{bonsai}.
% foraging working group

\subsection{The research the infrastructure, facility or resource will enable}

The hardware, software and user support we propose to distibute will allow
research groups around the world to develop their own long-duration and
naturalistic experiments, enabling a new type of animal experimentation.

These experiments are costly, since they require specialized arenas to house
freely behaving animals for extended periods of time, hardware to monitor their
behavior, to perform neural recording and to store large amounts of data. For
researchers not able to develop their own experiments, we will share openly the
behavioral and physiological recording from our mouse foraging experiments,
enabling them to study long-duration mouse foraing at the behavioral and
physilogical levels.

In addition, these foraging data will enable machine learning scientists around
the world to develop and apply ther own algorithms to the data shared in our
resource.

\subsection{Aims and objectives}

The aims of the proposed resource are:

\begin{enumerate}

    \item share behavioral and electrophysiolical recordings from long-duration
        and naturalistic foraging experiments

    \item share hardware and software specifications to build long-duration
        naturalistic experiments

    \item share machine learning methods, and their implementations, for online
        and offline processing the data generated by these experiments

    \item provide online advise to scientists interested in use the distributed
        data, hardware, software and methods.

\end{enumerate}

\subsection{Potential user communities, applications and benefits}

% # User communities
% . research groups interested in building long-duration and naturalistic experiments
% . ML data scientists
% . pharaceutical companies testing drug side effects on animal models
% . personalized medicine

