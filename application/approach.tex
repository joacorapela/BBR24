
The proposed resource will distribute behavioral and electrophysiological data
from our foraging experiments and machine learning methods to simulate and analyze
these data. We will initially contribute to the resource machine learning
methods that we have used for the analysis of our foraging data. We will later
invite method contributions from external method developers. To motivate these
contributions we will organize foraging data simulation and analysis
competitions.

\subsubsection{Long-duration and naturalistic foraging data}

We will distribute in the \href{https://dandiarchive.org/}{DANDI} archive data
and metdatata generated in our foraging experiments.

The data will include
video recordings from all cameras,
foraging wheels positions from all patches,
pellet delivery times from all patches,
mice weights recorded at the nest and
ultrasound recordings of mice vocalizations.

The metadata will include
subject information (birth data, sex and genetic strain) and
experiment information (start date, duration and type -- single or multiple
mice).

We will also provide Python routines to access specific data and meta data
items from the raw files.

The data generated by our experiments is very large. Storing one hour of
electrophysiological recordings requires xx gigabytes and one hour of video
recordings requires yy gigabytes. For this reason we will initially only share
publicly in the \href{https://dandiarchive.org/}{DANDI} archive a small number
of experimental sessions.
%
Investigating alternative methods to share the large amounts of data generated
by our experiments is a research item proposed for this resource.

\subsubsection{Methods to process behavioral measurements}

\paragraph{Linear dynamical systems models}

Linear dynamical systems (LDS) models are used to characterize time-varying
observations as a function of hidden (i.e., latent) continuous state variables that vary
over time. It assumes that both state and observed variables are Gaussian. The
Kalman filter algorithm can be used to infer the probability distribution of
latent variables given observations. More information about LDS can be found in
\citep[][part I]{durbinAndKoopman12}.

We have developed an implementation of LDS models and used it to infer denoised
positions, velocities and accelerations of foraging mice from noisy and
incomplete position measurements
\href{https://joacorapela.github.io/lds\_python/auto\_examples/tracking/plotFilterFWGMouseTrajectoryManualVsLearnedParams.html}{offline}
and
\href{https://bonsai-rx.org/machinelearning/examples/examples/LinearDynamicalSystems/Kinematics/ForagingMouse/README.html}{online}
with Bonsai.

\paragraph{Nonlinear and non-Gaussian dynamical systems models}

LDS models are very versatile and can be used to model a wide range of
time-series observations. However, there are cases where this model does not
apply and nonlinear and non-Gaussians extensions are needed, like extended
Kalman filter, the unscented Kalman filter and the particle
filter~\citep[][part II]{durbinAndKoopman12}.

We have created the Poisson Linear Dyanmical System model (PLDS) to characterize count
observations~\citep{mackeEtAl15} and distributed a Matlab
implementation\footnote{\url{https://bitbucket.org/mackelab/pop\_spike\_dyn}}.
This is an offline implementation, and we have not yet produced an online one.
We have neither use PLDS to model behavioral or neural foraging data.

\paragraph{Hidden Markov models}

The hidden Markov model (HMM) is a latent variable model similar to the LDS
model, but where the hidden states are discrete~\citep[][Chapter 13]{bishop06}. It is
used to assign discrete states to time series observations.

We have developed an offline
\href{https://github.com/joacorapela/hiddenMarkovModels}{implementation} of the
HMM, used it to find repeatable states in epileptic seizures of human subjects,
from Utah array recordings of their neural
activity~\citep{rapelaAndTodorovToBeSubmitted}. We have also used the HMM to
infer mouse foraging states from kinematic inferences from the LDS model. We
are currently building an online implementation of the HMM in Bonsai.

\paragraph{Switching linear dynamical systems models}

A switching linear dynamical system (SLDS) model is a hybrid/nonlinear system which
consists of several linear subsystems and a switching rule that decides which
of the subsystems is active at each moment in time \citep[Section
18.6]{murphy12}.

Over long periods of time the behavior of a mouse may alternate between
different states, where each state can be well modeled by a linear dynamical
system. This type of long-time behavior could be well modeled by a SLDS model.

We have no yet used SLDS models to characterize our foraging data.

\paragraph{Generalized linear models}

Generalized linear models (GLMs) are regression models for observations with diverse
noise distributions (i.e., noise distributions in the exponential family). For
example, they can be used to estimate a linear regression model with spike
count as the observation variable.

We have used a Gaussian GLM (i.e., a standard linear regression model) to
study how different regressors (e.g., average speed or acceleration
before entering a patch, amount of reward obtained in the previous visit to the
current patch, amount of reward obtained in the previous visit to the other
patch, weights) influence the length of a foraging bout in a short (three
hours) experimental session. The predictive power of this model was poor.

We have used online Bayesian linear regression models in Bonsai to estimate
receptive fields of cells in primary visual
cortex\footnote{https://ncguilbeault.github.io/machinelearning/examples/examples/LinearDynamicalSystems/LinearRegression/ReceptiveFieldSimpleCell/README.html}.
Here we use linear dynamical systems to perform online Bayesian linear
regression, which illustrates the versatility of linear dynamical systems.

\paragraph{Deep neural networks}

Deep neural networks are powerful nonlinear function approximators used in
supervised learning \citep{goodfellowEtAl16}. This networks require large
amounts of training data to achieve good performance. Thus, they are not good
models for data generated from conventional experiments generating small
datasets. But they are excellent models for large duration experiments such as
ours.

We plan to use them to predict the duration of mice foraging bouts from a
large set of regressors, as mentioned above, but using long experimental
sessions. The hope is that deep neural networks trained with large datasets
from our long-duration foraging experiments will overcome the limitations of
linear regression models and achieve excelent predictive power.

\paragraph{Multiple body parts and pose estimation methods}

Deep neural networks have been very successful for tracking animal body parts
in video recordings.
%
We have used DeepLabCut~\citep{mathisEtAl18} for tracking body parts in
single-animal experiments and SLEAP~\citep{pereriraEtAl22} for tracking body
parts in multiple-animal experiments. Other methods for animal body parts have
recently become available (e.g., multi-animal DeepLabCut~\citep{lauerEtAl22}
and lighting pose~\citep{bidermanEtAl23}).
%
We will apply these methods to our long-duration foraging recordins and report
comparisons of their performance.

\subsubsection{Methods to process neural measurements}

\paragraph{Latent variable models}

\begin{description}

    \item[Gaussian linear dynamical systems]

    \item[Poisson linear dynamical systems]

    \item[Gaussian process factor analysis]

    \item[Sparse variational Gaussian process factor analysis]

    \item[Latent factor analysis via dyamical systems (LFADS)]

    \item[CEBRA]

\end{description}

\subsubsection{Methods to simulate foraging data}

\subsubsection{Methods for compressing behavioral and neural measurements}

