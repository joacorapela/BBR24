
The proposed resource will distribute machine learning methods to process
behavioural data and neural data, to perform simulations and to compress
behavioral and neural data. The resource will also distribute data from our
foraging experiments that will be used to evaluate the functionality of all
methods distributed by us and of those contributed by external developers.

\subsubsection{Methods to process behavioral measurements}

\paragraph{Linear dynamical systems models}

Linear dynamical systems (LDS) models are used to characterize time-varying
observations as a function of hidden (i.e., latent) state variables that vary
over time. It assumes that both state and observed variables are Gaussian. The
Kalman filter algorithm can be used to infer the probability distribution of
latent variables given observations. More information about LDS can be found in
\citep[][part I]{durbinAndKoopman12}.

We have developed an implementation of LDS models and used it to infer denoised
positions, velocities and accelerations of foraging mice from noisy and
incomplete position measurements
\href{https://joacorapela.github.io/lds\_python/auto\_examples/tracking/plotFilterFWGMouseTrajectoryManualVsLearnedParams.html}{offline}
and
\href{https://bonsai-rx.org/machinelearning/examples/examples/LinearDynamicalSystems/Kinematics/ForagingMouse/README.html}{online}
with Bonsai.

\paragraph{Nonlinear and non-Gaussian dynamical systems models}

The LDS modes is very versatile and can be used to model a wide range of
time-series observation. However, there are cases where this model does not
apply and nonlinear and non-Gaussians extensions are needed, like extended
Kalman filter, the unscented Kalman filter and the particle
filter~\citep[][part II]{durbinAndKoopman12}.

We have created the Poisson Linear Dyanmical System model (PLDS) to characterize count
observations~\citep{mackeEtAl15} and distributed a Matlab
implementation\footnote{\url{https://bitbucket.org/mackelab/pop\_spike\_dyn}}.
This is an offline implementation, and we have not yet produced an online one.
We have neither use PLDS to model behavioral or neural foraging data.

\paragraph{Hidden Markov models}

The hidden Markov model (HMM) is a latent variable model similar to the LDS
model, but in the HMM states are discrete~\citep[][Chapter 13]{bishop06}. It is
used to assign discrete states to time series observations.

We have developed an offline
\href{https://github.com/joacorapela/hiddenMarkovModels}{implementation} of the
HMM, used it to find repeatable states in epileptic seizures of human subjects,
from unique Utah array recordings of their neural
activity~\citep{rapelaAndTodorovToBeSubmitted}. We have also used the HMM to
infer mouse foraging states from kinematic inferences from the LDS model. We
are currently building an online implementation of the HMM.

\paragraph{Generalized linear models}

Generalized linear models (GLMs) are regression models for observations with diverse
noise distributions (i.e., noise distributions in the exponential family). For
example, they can be used to estimate a linear regression model with spike
count as the observation variable.

We have used a Gaussian GLM (i.e., a standard linear regression model) to
study which of a large set of regressors (e.g., average speed or acceleration
before entering a patch, amount of reward obtained in the previous visit to the
current patch, amount of reward obtained in the previous visit to the other
patch, weights). We have used a Bonsai implementation of online Bayesian linear
regression to estimate receptive fields of cells in the primary visual
cortex.

\paragraph{Deep neural networks}

\paragraph{Switching linear dynamical systems models}

\paragraph{Pose estimation methods}

\subsubsection{Methods to process neural measurements}

\paragraph{Latent variable models}

\begin{description}

    \item[Gaussian linear dynamical systems]

    \item[Poisson linear dynamical systems]

    \item[Gaussian process factor analysis]

    \item[Sparse variational Gaussian process factor analysis]

    \item[Latent factor analysis via dyamical systems (LFADS)]

    \item[CEBRA]

\end{description}

\subsubsection{Methods to simulate foraging data}

\subsubsection{Methods for compressing behavioral and neural measurements}

\subsubsection{Data sharing}

